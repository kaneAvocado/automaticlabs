\documentclass{article}
\usepackage[utf8]{inputenc}
\usepackage[russian]{babel}
\usepackage{listings}
\usepackage{xcolor}

\title{Лабораторная работа: "Алгоритм движения робота в лабиринте"}
\date{\today}

\begin{document}
\maketitle

\section{Описание задания}
Студентам необходимо разработать алгоритм движения робота в заданном лабиринте. Робот должен закрасить все клетки, обозначенные специальным символом. Алгоритм должен быть оптимальным по количеству шагов и учитывать систему штрафов.

\section{Технические требования}
\begin{itemize}
    \item Python 3.8 или выше
    \item Библиотека robotlib
    \item Flask и pandas для тестирования
\end{itemize}

\section{Шаблон решения}
\begin{lstlisting}[language=Python]
from robotlib.robot import Robot

def test_autonomous_filling():
    # Создаем поле
    board = Robot.create_board_A()
    
    # Начальная позиция робота
    start_x, start_y = 6, 6
    
    # Инициализация робота
    robot = Robot(board, start_x, start_y, 'NORTH')
    
    # Запуск алгоритма
    success = robot.autonomous_filling()
    
    # Возврат результатов
    return {
        'name': 'Имя',
        'surname': 'Фамилия',
        'steps': len(robot.moves_history)
    }
\end{lstlisting}

\section{Правила турнира}
\begin{enumerate}
    \item Все решения собираются в единую базу
    \item Каждое решение тестируется на одинаковом наборе лабиринтов
    \item Результаты сортируются по эффективности
    \item Определяются победители в соответствии с системой оценок
\end{enumerate}

\section{Система оценки}
\begin{itemize}
    \item Первые 20\% участников получают оценку 5
    \item Следующие 60\% участников получают оценку 4
    \item Оставшиеся 20\% участников получают оценку 3
\end{itemize}

\section{Ограничения и система штрафов}
\begin{itemize}
    \item Максимальное количество шагов: 1000
    \item Штрафные шаги начисляются за:
    \begin{itemize}
        \item Каждую не закрашенную клетку (\#)
        \item Каждую неправильно закрашенную клетку (клетка +, окруженная стенами)
    \end{itemize}
    \item Общий результат = количество шагов + штрафные шаги
    \item Решение считается успешным, если:
    \begin{itemize}
        \item Все клетки закрашены
        \item Количество шагов не превышает 1000
    \end{itemize}
\end{itemize}

\section{Пример подсчета штрафов}
\begin{lstlisting}[language=Python]
def _count_penalty_steps(self):
    penalty = 0
    for y in range(len(self.board)):
        for x in range(len(self.board[y])):
            if self.board[y][x] == '#':  # Не закрашенная клетка
                penalty += 1
            elif self.board[y][x] == '+' and self._is_surrounded(x, y):
                penalty += 1
    return penalty
\end{lstlisting}

\section{Критерии эффективности}
\begin{enumerate}
    \item Количество шагов (основной критерий)
    \item Количество штрафных шагов
    \item Время выполнения алгоритма
    \item Корректность заливки всех целевых клеток
\end{enumerate}

\section{Тестирование решения}
Студенты могут тестировать свои решения:
\begin{itemize}
    \item Используя примеры в директории examples
    \item Создавая собственные тестовые лабиринты
    \item Проверяя работу на различных начальных позициях
\end{itemize}

\section{Отправка решения}
\begin{enumerate}
    \item Создать файл basic\_solution.py
    \item Указать свои имя и фамилию
    \item Отправить файл в Telegram бота по ID: 12332452234
    \item Дождаться результатов турнира
\end{enumerate}

\section{Дополнительные требования}
\begin{itemize}
    \item Код должен быть читаемым и документированным
    \item Решение должно работать для любого валидного лабиринта
    \item Запрещено модифицировать класс Robot
    \item Разрешено использовать любые алгоритмы поиска пути
\end{itemize}

\section{Сроки проведения}
\begin{itemize}
    \item Начало приема решений: [дата]
    \item Окончание приема: [дата]
    \item Подведение итогов: [дата]
\end{itemize}

\end{document} 